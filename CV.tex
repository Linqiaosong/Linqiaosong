\documentclass[12pt,a4paper,roman]{moderncv}        % possible options include font size ('10pt', '11pt' and '12pt'), paper size ('a4paper', 'letterpaper', 'a5paper', 'legalpaper', 'executivepaper' and 'landscape') and font family ('sans' and 'roman')

\moderncvstyle{classic}                             % style options are 'casual' (default), 'classic', 'banking', 'oldstyle' and 'fancy'
\moderncvcolor{black}                               % color options 'black', 'blue' (default), 'burgundy', 'green', 'grey', 'orange', 'purple' and 'red'
%\renewcommand{\familydefault}{\sfdefault}         % to set the default font; use '\sfdefault' for the default sans serif font, '\rmdefault' for the default roman one, or any tex font name

\usepackage[utf8]{inputenc}
\usepackage[scale=.75]{geometry}
%\setlength{\hintscolumnwidth}{3cm}
%\setlength{\makecvtitlenamewidth}{10cm}

\name{Qiaosong}{Lin}
\title{Wuhan University}
\address{College of Chemistry and Molecule Sciences}{Wuhan, Hubei, China, 430072}
\phone[mobile]{(86)-13627733161}
\email{linqiaosong@whu.edu.cn}
\social[Skype]{live:linqiaosong}
\photo[80pt][0.4pt]{qslin.jpg}                      
\quote{Current research interests include the persistent luminescent mechanism, time-dependent density functional theory calculation, and biological applications of persistent luminescence nanomaterials.}
\makeatletter\renewcommand*{\bibliographyitemlabel}{\@biblabel{\arabic{enumiv}}}\makeatother

\begin{document}
\makecvtitle

\section{Education}
\cventry{2017--present}{Bachelor's Degree (Junior Status)}{Wuhan University}{Wuhan, China}{}{GPA - 3.72/4.0 $\mid$ Major: Chemistry}

\section{Experience}

\subsection{\underline{Research Experience}}

\cventry{2018--present}{Quan Yuan's Group}{Key Laboratory of Analytical Chemistry for Biology and Medicine (Ministry of Education), Wuhan University}{Hubei, China}{}{
Research Interest:
\begin{itemize}
    \item Mechanism of long afterglow phenomenon
    \item Sythesis of inorganic long afterglow nanoparticles
    \item Application of long afterglow in biological analysis
\end{itemize}}

\cventry{2019 Jul.}{The Institute of Theoretical and Computational Chemistry}{Nanjing University}{Jiangsu, China}{}{Summer School of Theoretical and Computational Chemistry}

\cventry{2018 Jul.}{The National Center for Nanoscience and Technology}{}{Beijing, China}{}{Summer Exchange}



\subsection{\underline{Teaching Experience}}

\cventry{2019 Sept. --2020 Jan.}{Teaching Assistant}{College of Chemistry and Molecular Sciences}{Wuhan University}{}{Physical Chemstry I}

\cventry{2020 Feb.--Jun.}{Teaching Assistant}{College of Chemistry and Molecular Sciences}{Wuhan University}{}{Physical Chemstry II}

\cventry{2020 Feb.--Jun.}{Teaching Assistant}{College of Chemistry and Molecular Sciences}{Wuhan University}{}{Structural Chemstry A}


\section{Achievements}

\subsection{\underline{Publications}}

\cventry{[1]}{\underline{Lin, Q.};* Li, Z.;* Ji, C.; Yuan, Q.}{Electronic structure engineering and biomedical applications of low energy-excited persistent luminescence nanoparticles}{\textit{Nanoscale Adv.}}{\textbf{2020}, \textit{2}, 1380–-1394.}{*These authors contribute equally to this work.}

\cventry{[2]}{\underline{Lin, Q.}; Li, Z.; Yuan, Q.}{Recent advances in autofluorescence-free biosensing and bioimaging based on persistent luminescence nanoparticles}{\textit{Chin. Chem. Lett.}}{\textbf{2019}, \textit{30}, 1547–-1556.}{}

\cventry{[3]}{Wang, Y.;* Li, Z.;* \underline{Lin, Q.};* Wei, Y.; Wang, J.; Li, Y.; Yang, R.; Yuan, Q.}{Highly Sensitive Detection of Bladder Cancer-Related miRNA in Urine Using Time-Gated Luminescent Biochip}{\textit{ACS Sens.}}{\textbf{2019}, \textit{4}, 2124–-2130.}{*These authors contribute equally to this work.}

\cventry{[4]}{Qin, X.; \underline{Lin, Q.}; Yuan, Q.}{Applications of Upconversion Nanoparticles in Biological Diagnosis and Therapy}{\textit{Prog. Pharm. Sci.}}{\textbf{2019}, \textit{43}, 324–-333.}{}

\subsection{\underline{Other Achievements}}

\cventry{[1]}{Lin, Q.}{PyQTST Package}{https://github.com/Linqiaosong/PyQTST}{(2020)}{}

\cventry{[2]}{Fei, Y.; \underline{Lin, Q.}; Zhuang, L.}{Fermi-Softness Calculation Package}{https://github.com/idocx/q-e}{(2020)}{}

\cventry{[3]}{Lin, Q.}{QTST Tool}{https://github.com/Linqiaosong/QTST}{(2019)}{}



\section{Honor}
\cvitem{2018 Sept.}{\textbf{Scholarship of excellent students in Wuhan University (C)}, College of Chemistry and Molecular Sciences, Wuhan University.}
\cvitem{2018 Sept.}{\textbf{Outstanding Student Honor}, College of Chemistry and Molecular Sciences, Wuhan University.}
\cvitem{2019 Sept.}{\textbf{Scholarship of excellent students in Wuhan University (A)}, College of Chemistry and Molecular Sciences, Wuhan University.}
\cvitem{2019 Sept.}{\textbf{Merit Student Honor}, College of Chemistry and Molecular Sciences, Wuhan University.}
\cvitem{2019 Oct.}{\textbf{BlueMoon Corporation Scholarship}, Wuhan University.}
\cvitem{2020 Jul.}{\textbf{DICP Scholarship}, College of Chemistry and Molecular Sciences, Wuhan University.}

\section{Other Information}
\cvitem{\textbf{Computer Ability}}{C, C++, MATLAB, Python, Origin, Adobe Illustrator, Adobe Photoshop, \LaTeX}
\cvitem{\textbf{Calculation Software}}{Gaussian, ORCA, MRCC, Dalton, MOPAC, xTB, Multiwfn, VASP, CASTEP}
\cvitem{\textbf{Experience Ability}}{Inorganic synthesis, UV-Vis Spectroscopy, Fluorescence Spectroscopy, FTIR, XRD}

\section{Websites}
\cvitem{\textbf{Github}}{https://github.com/Linqiaosong}
\cvitem{\textbf{ORCID}}{https://orcid.org/0000-0003-4347-3361}
\cvitem{\textbf{RG}}{https://www.researchgate.net/profile/Qiaosong\_Lin}


\end{document}